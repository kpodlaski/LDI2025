\documentclass{beamer}%
\usetheme{Boadilla}%Berlin}
\usepackage[T1]{fontenc}
\usepackage[utf8]{inputenc}
\usepackage{lmodern}
\usepackage{amsmath}
\usepackage{amsthm}
\usepackage{amsopn}
%\usepackage{cite}
\usepackage{pdflscape}
\usepackage{afterpage}
\usepackage{changepage}
\usepackage{caption}
\usepackage{colortbl}
\usepackage{adjustbox}
\usepackage{hyperref}
\usepackage{caption}
\usepackage{subcaption}
\usepackage{tabularx}
\usepackage{multicol}
%\usepackage[style=authoryear]{biblatex}
%\bibliography{revbibl}
\usepackage{graphicx}
\usepackage{cite}
\usepackage{apacite}
\usepackage{outlines}
\setlength{\arrayrulewidth}{1pt}
\usepackage{tikz}
    \usetikzlibrary{positioning}
    \usetikzlibrary{decorations.pathreplacing}
    \usetikzlibrary{fadings}
    \usetikzlibrary{matrix}

\tikzset{basic/.style={draw,fill=blue!20,text width=1em,text badly centered}}
\tikzset{input/.style={basic,circle}}
\tikzset{weights/.style={basic,rectangle}}
\tikzset{functions/.style={basic,circle,fill=blue!10}}
\tikzset{hidden/.style={draw,shape=circle,fill=blue!30,minimum size=1.15cm}}
\tikzset{output/.style={draw,shape=circle,fill=red!20,minimum size=1.15cm}}

\newcommand\blfootnote[1]{%
  \begingroup
  \renewcommand\thefootnote{}\footnote{#1}%
  \addtocounter{footnote}{-1}%
  \endgroup
}

\makeatother
\setbeamertemplate{footline}
{
  \leavevmode%
  \hbox{%
  \begin{beamercolorbox}[wd=.2\paperwidth,ht=2.25ex,dp=1ex,center]{author in head/foot}%
    \usebeamerfont{author in head/foot}\insertshortauthor
  \end{beamercolorbox}%
  \begin{beamercolorbox}[wd=.8\paperwidth,ht=2.25ex,dp=1ex,center]{title in head/foot}%
    \usebeamerfont{title in head/foot}
    \insertshorttitle{}\hspace*{12ex}
    \insertframenumber/\inserttotalframenumber%\hspace*{1ex}
  \end{beamercolorbox}
  }%
  \vskip0pt%
}
\makeatletter
\setbeamertemplate{navigation symbols}{}

\theoremstyle{definition}
\newtheorem*{proposition}{Proposition}

\theoremstyle{definition}
\newtheorem*{enboxed}{}



\newcommand{\srcsize}{\@setfontsize{\srcsize}{5pt}{5pt}}

\AtBeginSection[]{
    \begin{frame}
    \vfill
    \centering
    \begin{beamercolorbox}[sep=8pt, center, shadow=true,rounded=true]{title}
    \usebeamerfont{title}\insertsectionhead\par
    \end{beamercolorbox}
    \vfill\end{frame}
}


\AtBeginSubsection[]{
    \begin{frame}
    \vfill
    \centering
    \begin{minipage}{0.6\textwidth}
     \begin{beamercolorbox}[sep=8pt, center, shadow=true,rounded=true]{title}
     \usebeamerfont{title}

        \begin{outline}
            \1[] \insertsectionhead
            \2 \insertsubsectionhead
        \end{outline}
    \end{beamercolorbox}
    \end{minipage}
    \vfill
    \end{frame}
}
\newcommand\colab{{\color{blue} COLAB \pause}} 
\hypersetup{colorlinks=true,linkcolor=blue,urlcolor=blue} 
%https://medium.com/@raycad.seedotech/convolutional-neural-network-cnn-8d1908c010ab

\begin{document}
\title{Jak sztuczna inteligencja widzi świat?}
\author[K. Podlaski]{Krzysztof Podlaski\\{\tiny podlaski@uni.lodz.pl}}
\institute[KSI UŁ]{ \tiny Katedra Systemów Inteligentnych\\Wydział Fizyki i Informatyki Stosowanej\\Uniwersytet Łódzki}

\date[5 XI 2025]{{\small Łódzkie Dni Informatyki,}\\ {\tiny Wydział Fizyki i Informatyki Stosowanej,\\ Uniwersytet Łódzki,\\
5 listopada 2025}}


\begin{frame}
\titlepage
\end{frame}

\begin{frame}{Agenda}
\tableofcontents
\end{frame}

\begin{frame}
  \frametitle{Materiały z zajęć}
  \begin{enboxed}
    Materiały znajdują się w sieci pod adresem:
    \url{https://github.com/kpodlaski/LDI2025}
  \end{enboxed}
\end{frame}

\section{Czym dla maszyny jest obraz}

\begin{frame}
  \frametitle{Digitalizacja obrazu}
  \begin{outline}
    \1 Obraz rzeczywisty,
    \1 Obraz zdigitalizowany \colab 
      \2 RGB
  \end{outline}
\end{frame}



\section{Elementy sieci Konwolucyjnych}
\subsection{Konwolucja}

\begin{frame}
  \frametitle{Operacja Konwolucji (Splotu)}
  \begin{outline}
    \1 Nakładanie na obraz filtrów
      \2 Zwanych często jądrami operacji
      \2 Kernel (ang.)
    \1 Filtry nakładane są fragmentami (oknami)
      \2 Okno porusza się nad obrazem 
        \3 Łącząc obraz źródłowy z filtrem
      \2 Otrzymujemy przetworzony obraz 
    \1 Przykład \href{https://en.wikipedia.org/wiki/Convolution}{animacja www} \pause \colab
    %https://medium.com/@RaghavPrabhu/understanding-of-convolutional-neural-network-cnn-deep-learning-99760835f148
  \end{outline}
\end{frame}

\begin{frame}
  \frametitle{Standardowe filtry}
  Dobrze znane filtry:
  \begin{outline}
    \1 Wygładzanie 
      \2 Smoothing, blur (ang.)
      \3 Box blur, gauss blur, ....
    \1 Wyostrzanie
      \2 Sharpen
    \1 Wykrywanie krawędzi
      \2 Edge detection (ang.) 
    \end{outline}
\end{frame}

\subsection{Operacja łączenia}
\begin{frame}
  \frametitle{Zmniejszanie rozmiaru, uogólnianie informacji}
  \begin{outline}
    \1 Poruszamy się oknem nad obrazem
      \2 Troszkę jak w konwolucji
      \2 Całe okno zamieniamy na jedną wartość \href{https://en.wikipedia.org/wiki/Pooling_layer}{www}\pause
    \1 Możliwe operacje
      \2 Zostawiamy wartość maksymalną
        \3 Max Pooling (ang.)
      \2 Zostawiamy wartość średnią
        \3 Ang Pooling (and.)
      \2 Zostawiamy wartość minimalną
        \3 Min Pooling (ang.) 
  \end{outline}
\end{frame}

\subsection{Neuron}
\begin{frame}
  \frametitle{Neuron (Perceptron)}
  \begin{outline}
    \1 Neuron jako operacja sumy ważonej
      \2 Dla każdego kanału wejścia
      \2 Przydzielamy wagę
      \2 Mając sygnał wejściowy sumujemy (wartość kanału wejścia)*(waga kanału)
    \1 Możemy nakładać próg aktywacji
      \2 Bias (ang.)
    \1 Na wyjście często nakładamy metodę aktywacji
      \2 Relu
      \2 Sigmoid
      \2 Tanh
    \1 \href{https://en.wikipedia.org/wiki/Artificial_neuron} {www}
  \end{outline}
\end{frame}


\section{Sieci neuronowe}
\begin{frame}
  \frametitle{Sieć warstwowa}
  \begin{outline}
    \1 Sieć neuronowa jednokierunkowa 
      \2 Ustawione kolejno warstwy
        \3 Wyjście warstwy jest sygnałem wejściowym dla następnej
      \2 Pierwsza warstwa przyjmuje sygnał wejściowy
        \3 Może to być obrazek
      \2 Wyjście z ostatniej warstwy jest wynikiem działania sieci neuronowej
    \1 Sieci uczy się wykonywania odpowiednich zadań
      \2 Najczęściej algorytm propagacji wstecznej lub jego pochodne
      \2 Zbiory danych
        \3 Uczący
        \3 Walidacyjny i Testowy
        \3 Zbiory powinny być rozłączne
  \end{outline}
\end{frame}

\begin{frame}
  \frametitle{Zadania sieci neuronowych}
  \begin{outline}
    \1 Klasyfikacja 
    \1 Regresja
    \1 Generowanie obiektu podobnego do wejścia
    \1 Generowanie tekstu (LLM)
  \end{outline}
\end{frame}

\begin{frame}
  \frametitle{Środowisko warszatatów}
  \begin{outline}
      \1 Logowanie:
        \2 użytkownik: {\color{orange} .$\backslash$student}
        \2 hasło: {\color{orange} student} \pause
      \1 Otworzyć katalog na pulpicie: {\color{blue} jupyter\_docker}
      \1 uruchomić: {\color{blue} run\_jupyter.cmd}
      \1 W przeglądarce otworzyć: \url{localhost:8888/lab?token=ldi2025}
  \end{outline}
\end{frame}

\begin{frame}
  \frametitle{Standardowa architektura sieci klasyfikującej obrazy}
  \begin{outline}
    \1 Sygnał wejściowy jest obrazkiem 
      \2 Bloki warstw
        \3 Konwolucyjna
        \3 Łącząca
      \2 Warstwy gęste
    \1 Przykład działania sieci CNN \url{https://poloclub.github.io/cnn-explainer}
  \end{outline}
\end{frame}
\end{document}

